% !TEX root = ../../main.tex
\chapter{Introduction}
\label{ch:intro}

Over the past four decades since the formulation of the Standard Model (SM), the theory has been tested with remarkable precision. 
Combining the theory of electroweak interactions with the theory of strong interactions in the same mathematical framework, the SM describes the physics of particles and fields up to the\,\TeV\, scale. All of the predicted constituents have been observed, culminating with the discovery of the Higgs boson in 2012~\cite{Higgs_atlas,Higgs_cms}.

Despite the success of the SM, it is not a complete theory of the physical universe. Notably, the SM does not incorporate a theory of gravity. Observed phenomena ranging from dark matter (inferred, for example, from the observed rotation curves of galaxies), to neutrino masses (inferred from the measurements of neutrino oscillations), to matter-antimatter asymmetry (the observed, extreme imbalance between the amount of particles and anti-particles in the universe) all indicate extensions of the current theory are needed. Among others, a critical theoretical concern is known as the hierarchy problem. An unsatisfying ``fine-tuning'' in the cancellation of divergent terms in the calculation of the Higgs mass is required to reproduce the observed value. 
% Mention weak vs gravity scale?

This thesis considers several models of new physics which purport to address shortcomings of the SM. Common to all of the theories considered is the prediction of as yet undiscovered, massive boson resonances which couple to the SM weak gauge bosons. If the masses of these new resonances are near the\,\TeV\,scale, they are kinematically accessible to searches at the Large Hadron Collider (LHC). Due to their decay to a pair of much lighter weak gauge bosons, the resonances would produce decay products that would be observed in the detector as highly boosted (large momentum) final state particles. 

In this search, the semi-leptonic final state refers to resonance decays where one of the daughter weak bosons decays to a lepton and a neutrino (leptonic decay), while the other decays to a pair of quarks (hadronic decay). For heavy resonances much more massive than the SM weak bosons, the two quarks from the boosted daughter weak boson decay will be highly collimated. The standard distance parameter used to reconstruct hadronic decays as ``jets'' will then be too large to separately distinguish the energy deposits of the two quarks, and too small to completely capture the total energy deposit from both. Consequently, in this search, a larger distance parameter is used to reconstruct the hadronically decaying boson as a single jet. By analyzing the substructure and mass of this jet, effective discrimination can be made between whether it is due to a true decay of a weak boson, or a processes imitating the decay with a similar topology.

% The search method
This thesis uses data collected in 2015 and 2016 with the ATLAS detector~\cite{ATLAS} at the LHC~\cite{LHC}, from $pp$ collisions with a center-of-mass energy $\sqrt{s}=13\,\TeV$, corresponding to a total integrated luminosity of $36.1\,\ifb$.  This search uses the mass distribution of the reconstructed diboson system as a discriminating variable to look for excesses above the SM background predictions. The results of the search are consistent with the SM background predictions. Constraints on the phase space of new physics are presented for selected benchmark signal models, and the results of the most recent searches~\cite{diboson_comb_2016, ATLAS_comb_run1, lvjj_run1, CMS_diboson_run2, CMS_diboson_run1run2, CMS_dijet_run1, CMS_diboson_run1} are significantly extended. The content and organization of this thesis are summarized below. 


% Organization of paper
In~\Ch{\ref{ch:stdmodel}}, a brief discussion of the SM of particle physics is given. The theoretical foundations of selected theories describing physics beyond the SM are given in~\Ch{\ref{ch:limitations}}, with an emphasis on the benchmark signal models considered in this thesis. In~\Ch{\ref{ch:lhc}} and~\Ch{\ref{ch:atlas}}, the experimental setup is outlined, including descriptions of the LHC, and the ATLAS detector, respectively. 
The reconstruction of physics objects from detector measurements in data and simulation is presented in~\Ch{\ref{ch:objreco}}. The strategy of the search is discussed in~\Ch{\ref{ch:analysisStrategy}}, including a discussion on the boosted event topology, the main SM backgrounds, and the methods of modeling signal and background processes. In~\Ch{\ref{ch:event_selection}}, the selection and classification of events in data and simulation are discussed. The signal regions sensitive to new physics, and the control regions used to validate the background modeling, are defined. 
The systematic uncertainties considered in this search are discussed in~\Ch{\ref{ch:syst}}. The statistical methods used to evaluate the results are outlined in~\Ch{\ref{ch:stats}}, and the final results are presented in~\Ch{\ref{ch:results}}. Finally, in~\Ch{\ref{sec:conclusions}} a summary of the search and a few concluding remarks are given.




