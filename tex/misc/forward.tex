% !TEX root = ../../main.tex
~\\[1in] % hack to put space at top.
\textbf{\Huge Foreward}\bigskip

\noindent 
%%%
%\iffalse
%%%

%% >> Arriving to NY as graduate student
In the fall of 2012, I joined the Columbia University Physics Department as a graduate student in New York. For the first two years of the program, I took classes and instructed undergraduate courses. 
%% >> Begin research with ADC
My first research experience took place during the summer of 2013 at Nevis Laboratories in Irvington, NY. I contributed to the testing of the Nevis12 ADC~\cite{Nevis12_paper}, designed to meet the requirements
%\footnote{The ADC should be radiation hard (both total ionizing dose and single event effects), dynamic range of approximately 12 bits, latency $<$ 200 ns, sampling at 40 MHz and have power consumption $<$145 mW/channel.} 
for the new Front End Boards (FEB) in the Liquid Argon (LAr) calorimeter Phase-I Upgrade to the ATLAS detector.
%\footnote{The LAr Phase-I upgrade is scheduled to take place during the second Long Shutdown (LS2) in 2018 to provide higher granularity and higher-resolution, moving the architecture from ``Trigger Towers" to ``Super Cells".} 
%The Nevis12 chip is a dual-channel 12-bit ADC including 4 stages of 1.5-bit(3 outputs) Multiplying Digital to Analogue Converters (MDAC), which allow for digital error correction. I helped create and implement an automated procedure to calibrate the MDACs. 
The Nevis12 chip is a dual-channel 12-bit ADC with digital error correction. I helped create and implement an automated calibration procedure, with which it was possible to quickly characterize the performance of the prototype chips, and measure the Effective Number of Bits (ENOB). In order to verify the radiation-hardness, we submitted the Nevis12 chips to radiation testing at Massachusetts General Hospital using their proton beam. The automated calibration procedure allowed for remote recalibration during irradiation, in order to accurately measure the ENOB as a function of dose, and to monitor the stability of the calibration constants. 

%% >> Work with LAr Operations
%% >> noisy cell application
Upon finishing coursework, I relocated to Geneva, Switzerland in June of 2014 to begin full-time research at the Large Hadron Collider (LHC), with the ATLAS experiment. Graduate students working with the ATLAS Experiment are required to perform a body of ``service work'' which aides in the improvement and continuation of the experiment. To satisfy this requirement, I joined the LAr calorimeter operations team, and originally focused on improving the online software. During my time with the online team, I created and developed an automated application which monitors electrically noisy calorimeter cells during data taking, and makes decisions about disabling the cells if the noise level is persistently high. Electrically noisy cells contribute to an overestimation of deposited energy. During the previous LHC run (Run-1), this process was done manually by on-call experts. Often, coordination between multiple groups was required to retrieve all of the appropriate information. Furthermore, it is only possible to identify groups of cells which could be responsible for the increased noise, so experts were required to use trial and error in a tedious, time-consuming manner. The current application pools together the rate information automatically, queues commands to disable cells in the noisy region, and evaluates the effect of disabling a given cell in a quick, systematic procedure. The shifter in the ATLAS control room is able to monitor all of the actions, and notify the on-call expert in case additional help is needed, greatly reducing the overall workload. 

%% >> Run Coordinator
After gaining hands-on experience with the online team and familiarizing myself with the LAr calorimeter operations, I took on greater responsibilities by serving as a LAr Run Coordinator. As Run Coordinator, I was required to be ``on-call'' around the clock for week-long shifts. A Run Coordinator is responsible for leading the operations of the subdetector: they head daily subdetector meetings to prioritize and plan the schedule, they present the status of the subdetector at daily ATLAS meetings, they coordinate tests and calibrations with other subsystems, they coordinate when hardware interventions are to be executed, and they are in contact with the shifters in the ATLAS control room and the LAr on-call team. The Run Coordinator is the first contact in case of any problems reported from the control room, and is responsible for quickly deciding the seriousness of the issue, how to resolve it, and who must be contacted. 

%% >> Work with Online Timing
Upon completion of my service work, I began interacting more closely with the diboson resonance analysis team, the search on which this thesis is based.
However, I continued contributing to LAr efforts. I was given the chance to present the status and performance of the LAr calorimeter during 2015 proton collisions at the 2016 LHCC meeting~\cite{lhcc_poster}. I was also asked to present a talk concerning the Phase-I and Phase-II upgrades of the LAr calorimeter at ICHEP in 2016~\cite{ryne_ichep_talk}. I performed additional studies for the LAr calorimeter by measuring and analyzing both the online and offline timing performance. Working with Kalliopi Iordanidou, we used 2015 $pp$ collision data to measure and provide corrections to the FEB fine-delays, in order to synchronize the online timing. In each subdetector of the calorimeter, the FEB timing showed excellent alignment: Gaussian fits yielded mean values well below 1 ns\footnote{
    0 ns corresponds to a highly relativistic particle originating from the center of the detector.
} and RMS values around 0.2 ns~\cite{LArCaloPublicResults2015}. Well grouped FEB timing below 1 ns ensures accurate energy reconstruction.
%\footnote{The sets of coefficients used to calculate the energy from the signal pulse samples are derived in bins of 25/24 ns, hence online timing better than $\sim$1 ns has a negligible effect on the energy.}. 

%% >> Work with Offline Timing
Following a study of 2011 and 2012 $pp$ data done by a previous Columbia student, Nikiforos Nikiforou, 
%~\cite{LArTiming_2012}, 
I optimized the calorimeter offline timing performance for the 2015 $pp$ data~\cite{LArTiming_2015}. A seven-step calibration procedure was used to significantly improve the time resolution and synchronization of the LAr calorimeter. Among other effects, the calibration takes into account energy dependence, cross-talk effects, and variations between runs. In the end, average calorimeter cell times centered to 0 ns from original offsets of $500-900$ ps, and time resolution improved from approximately $500-800$ ps to $200-270$ ps. Sub-nanosecond time resolution is important for various exotic physics searches involving long-lived neutral particles. To facilitate these searches, I coded a tool that applies the precision offline timing corrections derived from the study. Additionally, the tool can smear the timing of Monte Carlo generated events to match the measured time resolution in the detector. Notably, I utilized this tool in a cross-check study of the timing correlation between photons in the 2015 diphoton resonance search~\cite{diphoton_paper}.


%% Todo
%Summer school poster?

%%%
%\fi
%%%






