% !TEX root = ../../main.tex
\chapter{Conclusions}
\label{sec:conclusions}
% NWA -> width of 5\,\% is comparable to experimental mass resolution
% re-define abbreviations?

In this thesis, a search for new diboson resonances decaying to a boosted semi-leptonic final state ($WV\ra\ell\nu qq$) is presented, using 36.1\,\ifb\, of $pp$ collision data at a center-of-mass energy of $\sqrt{s}=13\,\TeV$, recorded in 2015 and 2016 by the ATLAS detector at the LHC. The data recorded represents the highest center-of-mass energy, instantaneous luminosity, and integrated luminosity in a hadron collider to date. The observed data is compatible with the SM background prediction, with no significant excesses observed. The search significantly extends the excluded mass ranges for several benchmark signal models, with respect to the most recent results~\cite{diboson_comb_2016, ATLAS_comb_run1, lvjj_run1, CMS_diboson_run2, CMS_diboson_run1run2, CMS_dijet_run1, CMS_diboson_run1}.

% Small discussion on theoritcal motivation, summary of the signal models
New diboson resonances are predicted by many BSM theories to help explain shortcomings of the SM. Theories of an extended Higgs sector predict new scalar resonances, and can provide mechanisms for CP violation, candidates for Dark Matter, and relax the constraints of the SM Higgs sector\footnote{ 
In particular, needing to explain both the non-zero masses of the weak gauge bosons, and the fermionic masses through {\em ad hoc} Yukawa couplings. 
}. A neutral, heavy Higgs (NWA) signal is used as a benchmark model for spin-0 resonances decaying to $WW$, with either ggF or VBF production.  
New spin-1 resonances are general predictions of models with an extended gauge sector or a new strongly interacting sector, and can offer explanations for charge quantization and the fine tuning problem.  
% Say what these explain
%, including GUTs, theories of extra dimensions, and composite Higgs models. 
An HVT model with $W'$ and $Z'$ resonances is used to characterize charged and neutral spin-1 resonances produced via qqF or VBF, using a simplified, phenomenological Lagrangian.  Two models are considered: model-A ($g_v=1$), which is representative of an extended gauge group, and model-B ($g_v=3$), which is representative of a composite Higgs model with suppressed fermionic couplings. A neutral spin-2 resonance is predicted by Randall-Sundrum (RS) theories of warped extra dimensions, offering a geometrical explanation for the hierarchy problem. The first Kaluza-Klein excitation of the bulk RS graviton is considered with ggF production, for $k/\overline{M}_{\rm Pl}=1.0$ and $k/\overline{M}_{\rm Pl}=0.5$. 

The search focuses on the so-called boosted regime, for resonance masses above $500\,\GeV$. In this kinematic region, the mass resolution is improved through the use of large-R jets ($R=1.0$) to reconstruct the boosted weak boson from the highly collimated quarks in the hadronic decay ($V\ra q\bar{q}'/q\bar{q}$ for $q,q'=u,d,c,s,b$). The mass resolution is further improved through the novel use of the combined mass, taking into account both calorimeter-based measurements and track-assisted-based measurements. Requiring one of the boosted weak bosons to decay leptonically ($W\ra\ell\nu$ for $\ell=e,\mu$) offers a compromise between the clean and efficient reconstruction of high-energy leptons in the noisy QCD background of a hadron collider, and the larger branching ratio of the weak bosons' decays to quarks. 

The sensitivity to new physics is improved from previous iterations of the search through further optimizations of the event selection. A new multijet cut is implemented to remove a small QCD contamination in the high-$\pT$ region of the $e$-channel. The signal region definitions are adapted to include both 50\,\% and 80\,\% efficiency working points of the large-R jet boson tagger, increasing the signal acceptance times efficiency of the search.  New signal regions are included through the addition of a VBF selection, in order to set independent limits on the selected benchmark signal models based on their production mechanism. 


Upper limits on cross section times branching ratio to $WV$ are calculated as a function of resonance mass for the selected benchmark signal models at 95\,\% CL. The simulated signal samples are generated separately for VBF, and ggF or qqF production. Signal mass points above 500\,\GeV\, are tested, up to 3\,\TeV\, for scalar models, and up to 5\,\TeV\,(4\,\TeV) for spin-1 and spin-2 models produced via ggF or qqF (VBF, spin-1 model only). For the HVT signal model with qqF production, $W'$ resonances are excluded at 95\,\% CL for masses below $2.80\,\TeV$\, and $2.99\,\TeV$\, for model-A and model-B, respectively; while $Z'$ resonances are excluded at 95\,\% CL for masses below $2.73\,\TeV$\, and $3.00\,\TeV$ for model-A and model-B, respectively. A bulk RS graviton resonance produced via ggF is excluded for masses below $1.75\,\TeV$\, for $k/\overline{M}_{\rm Pl}=1.0$ at 95\,\% CL. For $k/\overline{M}_{\rm Pl}=0.5$, masses below $0.98\,\TeV$\, and masses between $1.02-1.35\,\TeV$\, are excluded at 95\,\% CL.  

With respect to the previous results~\cite{diboson_comb_2016,CMS_diboson_run2}, the excluded mass range is extended by 450\,\GeV\, (390\,\GeV) for an HVT $W'$ resonance with model-A (model-B), 380\,\GeV\, (400\,\GeV) for an HVT $Z'$ resonance with model-A (model-B), and 650\,\GeV\, for a bulk RS graviton with $k/\overline{M}_{\rm Pl}=1.0$. The first excluded mass range for a bulk RS graviton with $k/\overline{M}_{\rm Pl}=0.5$ is presented. Stringent upper limits are set for a scalar, heavy Higgs resonance (NWA), and for $W'$ and $Z'$ resonances in an HVT model produced via VBF. 

As the LHC continues operation, the search for new diboson resonances remains a vital probe of physics beyond the SM. After a planned upgrade in 2019-2020, Run-III will feature an increase in the center-of-mass energy (up to $\sqrt{s}=14\,\TeV$), an increase in the instantaneous luminosity ($\sim 2.0\times 10^{34}$ cm$^{-2}$s$^{-1}$), and an increase in the total integrated luminosity ($\sim 300\,\ifb$).  Further on the horizon, during the High Luminosity LHC phase ($\sim2026$), additional upgrades will allow for an instantaneous luminosity of up to seven times the nominal value, and facilitate the collection of up to $3000\,\ifb\,$of data.
These improvements will lend increased sensitivity to exceedingly rare processes in the continuing search for a more complete description of particle physics. 



